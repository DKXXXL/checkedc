\section{Introduction}\label{sec:intros}

Vulnerabilities due to memory corruption, especially spatial memory corruption, 
are still a major issue for C programs~\cite{cvetrend, microsoftmemsafe, Zeng:2013:SRF:2534766.2534798} 
despite many efforts that tried to prevent them~\cite{song2019sanitizing}.
Several industrial and research efforts, including CCured~\cite{Necula2005},
Softbound~\cite{softbound}, and ASAN~\cite{Serebryany2012},
have investigated ways to better compile C programs with automatic spatial safety enforcement.
These approaches all impose performance overheads deemed too high for deployment use. 
% 
%\liyi{Should we say this? Is the paper about overhead reducing? }
Recently,~\citet{Elliott2018} and~\citet{li22checkedc} introduced and formalized \checkedc, an
open-source extension to C,
to ensure a program’s spatial safety by introducing new pointer types,~\ie checked ($\cmode$) pointer types.
%
The checked pointers are represented as system-level memory words without ``fattening'' metadata~\cite{duck2016heap}, 
and ensuring backward compatibility,~\ie developers can use checked and regular (unchecked $\umode$) pointers within the same program.
%users are able to split code into checked and unchecked regions and incrementally convert
%C code in the unchecked region to Checked C code in the checked one.
However, as we explain in~\sect{subsec:nosafetyagsintuncheckedcode}, the
unconverted or unchecked ($\umode$) code can violate guarantees provided in $\cmode$
regions.
%
We need to ensure that~\emph{code executed as part of 
unchecked ($\umode$) regions does not lead to the safety violations in checked ($\cmode$) regions  with the use of program partitioning mechanism \cite{rul2009towards}}.

% One possible solution is Checked C
% \begin{itemize}
% \item Think of it as migratory/gradual typing, but for C.
% \item Distinct pointer types, but which are backward binary- and source-compatible.
% \item Checked regions, containing only checked pointers and restricted
%   idioms, aim to ensure spatial safety
% \item Implemented as an extension to Clang/LLVM. Good performance
%   (compared to ASAN etc.)
% \end{itemize}
\iffalse
\noindent
\textbf{No safety against unchecked code.} The \checkedc spatial safety guarantee applies to completely converted programs, i.e., programs that uses only checked types.
Although, the backward compatibility of Checked C helps a partially annotated program to enjoy spatial memory safety on those regions using only Checked pointers (i.e., checked or safe regions).
But, the unconverted code regions (or unsafe regions) can affect pointers in safe regions and violate certain assumptions leading to vulnerabilities, as demonstrated by cross-language attacks~\cite{mergendahlcross}.
Although the blameless proof exists~\cite{ruef2019achieving} for~\checkedc, it does not state that spatial safety violations cannot happen in Checked regions but rather states that Checked regions~\emph{cannot be blamed for any spatial safety violations}.
Consider the following example:
\begin{minted}[xleftmargin=25pt, mathescape, linenos, escapeinside=||, fontsize=\footnotesize{}]{c}
// Checked code
int func(array_ptr<char> p : count(5)) {
|\textcolor{red}{\faChainBroken}|..p[4]..
}
// unchecked code
...
str = "he";
...
|\textcolor{red}{\faBug}|assume_bounds_cast<char>(str, 5); 
...
char ptr[16];
...
len <- derived from user input
...
|\textcolor{red}{\faBug}| memcpy(ptr, buff, len); // buffer overflow
\end{minted}
Here, the checked function~\inlinecode{func} expected a pointer to a buffer of five elements, but unchecked code violated it and invoked the function with a buffer of 2 elements.
This results in a spatial safety violation (\textcolor{red}{\faChainBroken}) in the Checked region, but of course, the blame or the root cause is in the unchecked region (\textcolor{red}{\faBug}).
Furthermore, since checked and unchecked regions execute in the same address space, spatial memory corruptions in unchecked regions (Line 15) can take down the complete program despite having checked regions.
We need~\emph{an isolation mechanism to ensure that code executed as part of unchecked regions does not violate the safety guarantees in checked regions}.
\fi
% 

Existing such mechanisms are not suitable as they are based on process isolation and have high overhead, and are \emph{data-centric} (\sect{subsec:background:programpart}).
%Furthermore, these techniques are hard to engineer to co-exist with~\checkedc.
%\liyi{why? Can we not say this? It seems that the whole reason people should care our work is we did it in \checkedc, but \checkedc itself is the selling product here.... }
But in our case, we want a low-overhead code-centric partitioning, where the $\umode$ region code (or functions) should be isolated (or partitioned) from $\cmode$ one. We also want the technique to co-exist and be compatible with Checked C guarantees such that the partition containing $\cmode$ region code can maintain spatial safety.

Here, we propose a type-directed code-centric program partitioning approach.
Specifically, our system,~\systemname, extends \checkedc's checked and unchecked pointer types---representing safe and unsafe program pieces---with \textbf{tainted} (\taintt) types running on an isolated sandbox mechanism,
forbids the communication between checked and unchecked type entities, and enforces the communication between checked and unchecked types
through the uses of tainted types with additional validity checks. 
%which can be used to mark functions and pointers that need to be isolated from the original program, while checked types follow the standard \checkedc typing rules.
%However, our type-system disallows any unsafe interactions between taint and untainted types.
% 
% Specifically, in our system,~\systemname, extending CheckedC,
% functions and pointers that need to be isolated from the original program can be
% marked with ~\textbf{tainted} (\taintt) types.
% 

The developer starts by marking desired (\ie unchecked, $\umode$) functions and pointers used in functions as tainted. 
Then, \systemname partitions the given program into two partitions (\umode and~\cmode regions) of different privileges:

\begin{itemize}
\item~\umode \emph{region} (low privilege tainted region, extended from the unchecked region in \checkedc): this partition contains tainted types (\ie functions and pointers) and can only access tainted and unchecked pointers.
\item~\cmode \emph{region} or~\emph{safe region} (high privilege untainted or checked region): This partition contains the remaining (untainted) code and data and has complete access to~\cmode region.
The functions in~\cregion can invoke any function in~\ucregion and access all its data but not the other way around, except for call-back functions, which we will discuss later.
\end{itemize}

The $\cmode$ region code is executed as a regular program, while the~\ucregion partition will be executed in an existing sandboxed environment (\eg WASM sandbox), with additional instrumentations to facilitate the communication between code in~$\cmode$ and~$\umode$ regions.

The combination of tainted types and privileged partitions
enforces isolation and provides memory safety without transforming all unchecked C code to \checkedc code,
because unchecked types can stay in \ucregion, and \cregion code can access tainted type entities that are allocated 
in \ucregion.
% 
Although memory isolation prevents direct violations,~\ucregion code can still affect~\cregion through tainted pointers by confused deputy attacks~\cite{rajani2016access, machiry2017boomerang},~\eg by using a valid~\cregion address in a tainted pointer.
Our compiler avoids these attacks by ensuring using dynamic checks that tainted pointers validly point to~\ucregion address space.
Such checks are statically generated by our compiler.

In summary, we make the following three main contributions.

\myparagraph{\systemname Type System, Formalism and Compiler}
We present a type system that integrates tainted types with~\checkedc and provides additional guarentees---the~\emph{non-crashing} and \emph{non-exposure} guarantees,~\ie a well-typed~\systemname program can never crash due to spatial safety violations,
as well as \ucregion code cannot directly observe a checked pointer address.
%no \cregion pointer addresses will be leaked in
%\ucregion code
We extend the \checkedc compiler to support the type system and
formalize it by extending~\checkedc formalism~\citet{li22checkedc} with the non-crashing and non-exposure guarantees.
We formally prove theorems related to the two guarantees and use model-based randomized testing (CITE) to certify the simulation relation between the~\systemname semantics and its compiler formalism.
To the best of our knowledge,~\systemname is the first C(-like) language and compiler formalism with the program partitioning mechanism.

\myparagraph{Type-Directed Program Partitioning}
We present a type-directed program partition technique to separate $\cmode$ and $\umode$ code regions and ensure the above guarantees.
%Our type system restricts checked pointers usage only in~\cregion and ensures that~\ucregion can only access tainted pointers.
Our modular design enables us to use existing sandbox techniques to enforce memory and execution isolation, with the implementation of tainted pointers in the \systemname compiler.

\myparagraph{Supporting callbacks to~\cregion with no Checked Pointer Exposure}
Although we disallow access to~\cregion from~\ucregion directly, there can be cases where such access is needed.
Specifically, when~\cregion wants to provide access to certain shared checked data to~\ucregion.
To enable this, we support callback functions in~\cregion that can be invoked from~\ucregion through function pointers.
%However, knowing the address of~\cregion functions in~\ucregion violates the program partition principle of \systemname.
However, knowing the address of~\cregion functions in~\ucregion violates the non-exposure guarantee and leads to other attacks~\cite{hauser2019sleak}.

We handle this by using indirection. Specifically, instead of directly accessing the~\cregion callbacks, the ~\ucregion accesses them using a tainted-typed protected trampoline function, which directs the execution to the appropriate callback function.
In addition, the trampoline function itself is referenced using an opaque index rather than its virtual address, implemented through
existing sandboxing techniques.
% 
%In our \systemname formalism, we formally verified that our formalism is
%\emph{non-exposure}, \ie, no \cregion pointer addresses will be leaked in
%\ucregion code.

\ignore{

-- NEED TO FINISH THIS--
Our third contribution is an added-up feature
to support checked (function) pointer callbacks in unchecked code regions.
When designing a multi-threaded system, users might want 
to provide a third party interface that allows third party developers to create new program features, while keeping these programs in unchecked code regions.
Moreover, they do want to provide them a (function) pointer pointing to checked data fields.

However, accessing a checked pointer in an unchecked region violates the program partition principle of \systemname.
To resolve the conflict, we develop two mechanisms in \systemname 
and maintain a stronger \textit{non-exposure} guarantee on of the non-crashing guarantee; 
that is, no checked pointer addresses can be observed in an unchecked code region.
The first mechanism allows nested checked and unchecked code regions.
Users can context switch between checked and unchecked code regions 
by nested using the keywords $\echeckedtext$ $\euncheckedtext$.
The type system ensures that no checked pointers can be accessed across the context switching.
The second one is that a call to a checked pointer in unchecked code regions 
must be surrounded by a \textit{tainted shell}; 
i.e., a tainted function pointer that points to a checked region possibly holding checked pointers.
In this case, no checked pointer address will be observed in the unchecked code regions.
% 
\mzu{In \systemname, for every checked function, 
we automatically compile a tainted version by surrounding the function without a
tainted shell.}
\mz{What does this mean?}
}
% 

\iffalse
\myparagraph{Formalizing the Type System, Semantics and Compiler}
%
We developed a core formalism named~\lang, which extends
\citet{li22checkedc}~\checkedc formalism with the non-crashing guarantee and other new features below.
We formally prove the~\emph{non-crashing theorem},~\ie 
a well-typed~\lang program can never crash due to spatial safety violations.
We utilize the model-based randomized testing (CITE) to certify the simulation relation between the~\lang semantics and the compiler.
Specifically, we use a conversion tool that converts expressions from~\lang into actual~\checkedc code that can be compiled by the~\checkedc compiler. We create a random program generator 
based on the typing rules of~\lang and ensure that~\lang and~\checkedc compiler are consistent after conversion, both statically and dynamically.  
To the best of our knowledge,~\lang is the first C(-like) language and compiler formalism with the program partitioning mechanism.
%To the best of our knowledge, \systemname is the first work of formalizing C function pointers with security guarantee.
\fi

We evaluated~\systemname by partioning seven large real-world programs to demonstrate its effectiveness.
Our evaluation shows that~\systemname provides a flexible, low-overhead program partitioning mechanism and guarantees spatial memory safety.

\ignore{
\aravind{Fix the following text after all the sections of the papers are finalized.}
\liyi{the following might be removed if space is needed, since it is usually not quite useful.}
We begin with a review of \checkedc{} \mzr{and introduction of} new features in
\systemname (Section~\ref{sec:overview}), present our main contributions
(Sections~\ref{sec:formal}--\ref{sec:evaluation}), and conclude with a
discussion of related and future work (Sections~\ref{sec:related},
\ref{sec:conclude}). All code and proof artifacts (both for Coq and Redex) can
be found at \url{https://github.com/plum-umd/checkedc}.
}

\ignore{
\noindent
\textbf{Converting C to Checked C.} The safety guarantees of Checked C come with certain restrictions. For instance, as shown below, Checked C programs cannot use address-taken variables in a bounds expression as the bounds relations may not hold because of possible modifications through pointers.
\begin{minted}[xleftmargin=30pt, mathescape, escapeinside=||, fontsize=\footnotesize]{c}
...
array_ptr<int> p : count (n) = NULL;
|\textcolor{red}{\faTimes}|..,&n,.
\end{minted}
Consequently, converting existing C programs to Checked C might require refactoring, e.g., modifying the above program to not use~\inlinecode{&n} expression, which might require considerable effort~\cite{duanrefactoring} depending on the program's complexity. 
Recently, Machiry et al. developed~\threec~\cite{machiry2022c} that tries to automatically convert a program to Checked C by adding appropriate pointer annotations.
However, as described in \threec, complete automated conversion is infeasible and requires the developer to convert some code regions manually.
Although, the backward compatibility of Checked C helps a partially annotated program to enjoy spatial memory safety on those regions using only Checked pointers (i.e., checked or safe regions).

\noindent
\textbf{No safety against unchecked code.} But, the unconverted code regions (or unsafe regions) can affect pointers in safe regions and violate certain assumptions leading to vulnerabilities, as demonstrated by cross-language attacks~\cite{mergendahlcross}.
Although the blameless proof exists~\cite{ruef2019achieving}, it does not state that spatial safety violations cannot happen in Checked regions but rather states that Checked regions~\emph{cannot be blamed for any spatial safety violations}.
Consider the following example:
\begin{minted}[xleftmargin=25pt, mathescape, linenos, escapeinside=||, fontsize=\footnotesize{}]{c}
// Checked code
int func(array_ptr<char> p : count(5)) {
|\textcolor{red}{\faChainBroken}|..p[4]..
}
// unchecked code
...
str = "he";
...
|\textcolor{red}{\faBug}|assume_bounds_cast<char>(str, 5); 
...
char ptr[16];
...
len <- derived from user input
...
|\textcolor{red}{\faBug}| memcpy(ptr, buff, len); // buffer overflow
\end{minted}
Here, the checked function~\inlinecode{func} expected a pointer to a buffer of five elements, but unchecked code violated it and invoked the function with a buffer of 2 elements.
This results in a spatial safety violation (\textcolor{red}{\faChainBroken}) in the Checked region, but of course, the blame or the root cause is in the unchecked region (\textcolor{red}{\faBug}).
Furthermore, since checked and unchecked regions execute in the same address space, spatial memory corruptions in unchecked regions (Line 15) can take down the complete program despite having checked regions.
We need~\emph{an isolation mechanism to ensure that code executed as part of unchecked regions does not violate the safety guarantees in checked regions}.

This mechanism is called program partitioning~\cite{rul2009towards}, and there has been considerable work~\cite{tan2017principles, brumley2004privtrans, bittau2008wedge, lind2017glamdring, liu2017ptrsplit} in the area. Most of these techniques are~\emph{data-centric}~\cite{lind2017glamdring, liu2017ptrsplit}, wherein program data drives the partitioning. E.g., Given sensitive data in a program, the goal is to partition functions into two parts or partitions based on whether a function can access the sensitive data.
The performance overhead of these approaches is dominated by marshaling costs and depends on the usage of sensitive data.
The overhead of state-of-the-art approaches~\cite{lind2017glamdring, liu2017ptrsplit} is prohibitive and varies from 37\%-163\%.
But in our case, we want a low-overhead code-centric partitioning, where the unchecked code (or functions) should be isolated (or partitioned) from checked code. We also want the technique to co-exist and be compatible with Checked C guarantees such that the partition containing checked code should still enjoy its spatial safety.

In this work, we propose a type-directed code-centric program partitioning approach.
Specifically, our system,~\systemname, extends Checked c using~\textbf{tainted} (\taintt) types, which can be used to mark functions and pointers that need to be isolated from the original program.
The tainted types long with Checked C allow annotating pointer along two dimensions, i.e., (i) taintedness: a pointer can be either tainted or not (untainted), and (ii) checkedness: it can be either checked or not.
The checked types follow the standard Checked C typing rules. However, our type-system disallows all interactions between taint and untainted types.
The developer starts by marking desired (unchecked or risky) functions and pointers used in these functions as tainted.
Second,~\systemname partitions the given program into two partitions (\ucregion and~\cregion)  of different privileges:
\begin{itemize}
\item~\ucregion (low privilege tainted region): This partition contains tainted types (i.e., functions and pointers) and can only access tainted pointers.
\item~\cregion (high privilege untainted region): This partition contains the remaining (untainted) code and data and has complete access to~\cregion{}.
The functions in~\cregion can invoke any function in~\ucregion but not the other way around, except for call-back functions, which we will discuss later.
\end{itemize}
Finally, during program execution, the~\ucregion partition will be executed in an existing sandboxed environment (e.g., WASM sandbox), and our compiler will add the necessary instrumentation to facilitate the communication between code in~\cregion{} and~\ucregion{}.

The combination of tainted types and privileged partitions enables us to enforce isolation and provide memory safety without marshaling costs.
As functions in the~\ucregion can only access tainted types,~\cregion functions should use tainted types to pass pointer arguments to~\ucregion functions. 
We avoid marshaling by allocating all tainted pointers (i.e., tainted buffers) in~\ucregion and thus can be accessed in both partitions.
Although memory isolation prevents direct violations,~\ucregion code can still affect~\cregion through tainted pointers by confused deputy attacks.
Our compiler avoids these attacks by ensuring, either statically or through dynamic checks, that tainted pointers can only point to~\ucregion address space.

%In this work, we propose~\systemname, an isolation mechanism that enables unchecked regions or, in general, any arbitrary set of functions to be executed in an isolated address space (\ucregion).
%So that checked regions can still enjoy true spatial memory safety even in the presence of unchecked or unsafe regions.
%\systemname extends Checked c using~\textbf{tainted} (\taintt) types, representing the code and data that belong to the~\ucregion.
%
%\noindent
%\textbf{Automated Partitioning.} A pointer (checked or unchecked) can be marked as tainted; similarly, a function can be marked as tainted.
%Along with enforcing spatial safety constraints of checked tainted pointers, the~\systemname compiler will bundle all the tainted types (functions and pointers) so that they remain in an isolated sandbox (\ucregion), and non-tainted types will remain in the process address space (\cregion).
%
%\noindent
%\textbf{Memory Access Restrictions.} The functions in the~\ucregion can only access tainted pointers, i.e., memory within the corresponding sandbox region.
%However, the functions in~\cregion have no-such restrictions and can access both c- and uc-region's memory.
%Similarly, the functions in~\cregion can invoke any function in~\ucregion but not the other way around, except for call-back functions, which we will discuss later.
%
%\noindent
%\textbf{Marshaling free communication.} As functions in the~\ucregion can only access tainted types,~\cregion functions should use tainted types to pass pointer arguments to~\ucregion functions. 
%We avoid marshaling as all tainted buffers (i.e., regions pointed by tainted pointers) are allocated in~\ucregion and thus can be accessed in both c- and uc-region.
%
%\noindent
%\textbf{Additional checks.} Although memory isolation prevents direct violations,~\ucregion code can still affect~\cregion through tainted pointers by confused deputy attacks.
%However, our compiler avoids these attacks by ensuring, either statically or through dynamic checks, that tainted pointers in~\cregion can only point to~\ucregion address space, and our type system enforces that tainted pointers cannot be assigned to non-tainted pointers.
%
%%\iffalse
%Recently, Narayan et al. proposed RLBox, which enables sandboxing libraries by making data- and control-flow at the sandbox interface explicit, using types.
%Specifically, RLBox defines a tainted type (\code{tainted<...>}), which is a C++ template class and can be used to have any type tainted (e.g.,~\code{tainted<int>},~\code{tainted<float*>}, etc).
%The tainted types represent values or pointers to buffers that can be accessed from a sandbox.
%The program or unsandboxed code can use only tainted type variables to communicate (i.e., parameters and return values) with sandboxed functions.
%Although RLBox provides a practical technique for sandboxing libraries, it has the following drawbacks:
%\begin{itemize}
%\item\textbf{Marshalling Cost.} Any non-scalar value (e.g., a pointer to a struct) that need to be passed to a sandboxed function should be first copied into a tainted buffer (\emph{marshalling}). Then the corresponding tainted value (i.e., tainted pointer) should be passed to the sandboxed function.
%This can lead to considerable overhead for large objects.
%\item\textbf{Developer Written Sanitization.} The unsandboxed code can access tainted type variables only through certain special (i.e., sanitization) routines.
%This forces developers to sanitize every value that comes from sandboxed library functions.
%However, the correctness of the sanitization is entrusted to the developer.
%Consequently, any mistake in sanitization methods can still lead to vulnerabilities.
%\item\textbf{Library Level Sandboxing.} The unit of sandboxing in RLBox is the whole library. However, we want to sandbox a set of functions in a program.
%\end{itemize}
%\fi

% How to highlighgt code in minted
% highlightlines={1-3}, highlightcolor=red
%\ignore{
\begin{listing}[t!]
  \begin{tabular}{c c}
    \begin{minipage}[b]{.22\textwidth}
\inputminted[mathescape, escapeinside=||, fontsize=\tiny{}]{c}{examples/orig1.c}
    \end{minipage} &
    \begin{minipage}[b]{.22\textwidth}
\inputminted[mathescape, escapeinside=||, fontsize=\tiny{}]{c}{examples/orig2.c}
    \end{minipage} \\
   (a) Original C code & (b) After initial conversion.\\
  \end{tabular}
\caption{(Contrived) Example demonstrating various phases of.}
\label{lst:comb}
\end{listing}

\begin{listing}[t!]
\inputminted[mathescape, escapeinside=||, fontsize=\tiny{}]{c}{examples/firstrun.c}
\caption{(Contrived) Example demonstrating various phases of.}
\label{lst:initialconv}
\end{listing}

\begin{listing}[t!]
\inputminted[mathescape, escapeinside=||, fontsize=\tiny{}]{c}{examples/humanannotations.c}
\caption{(Contrived) Example demonstrating various phases of.}
\label{lst:humantaint}
\end{listing}

\begin{listing}[t!]
\inputminted[mathescape, escapeinside=||, fontsize=\tiny{}]{c}{examples/humanadjustments.c}
\caption{(Contrived) Example demonstrating various phases of.}
\label{lst:humanadjust}
\end{listing}
%}
}



%\begin{figure}
\begin{center}
\begin{tikzpicture}[scale=0.75]
%\node[text width=3cm] at (1.5,-0.5) 
%    {some text spanning three};
%\draw[step=5cm,gray,very thin] (0,0) grid (5,5);


% This is all not protected region.
%Top left: High risk and low safety
\fill[red!50!white] (0,2.5) rectangle (2.5,5);
% bottom left: low risk and low safety
\fill[red!20!white] (0,0) rectangle (2.5,2.5);

% This is all protected.
% Top right: High risk and high safety
\fill[green!40!white] (2.5,2.5) rectangle (5,5);
% bottom right: low risk and high safety
\fill[green!40!white] (2.5,0) rectangle (5,2.5);


% High risk, high safety
\Text[x=3.8cm,y=5.3cm,fontsize=\footnotesize]{\tiny Checked C}
\Text[x=3.8cm,y=3.5cm,fontsize=\footnotesize]{\tiny Functions with spatial}
\Text[x=3.8cm,y=3.3cm,fontsize=\footnotesize]{\tiny safety for all}
\Text[x=3.8cm,y=3.1cm,fontsize=\footnotesize]{\tiny pointers.}

\Text[x=1.2cm,y=5.3cm,fontsize=\footnotesize]{\tiny No Safety}
\Text[x=1.2cm,y=4.3cm,fontsize=\footnotesize]{\tiny Sanboxed Functions.}
\Text[x=1.2cm,y=4.1cm,fontsize=\footnotesize]{\tiny (Tainted function)}

\Text[x=1.9cm,y=2.3cm,fontsize=\footnotesize]{\tiny Checked C}


%\Text[x=3.8cm,y=2.1cm,fontsize=\footnotesize]{\tiny Retrofitting}
%\Text[x=3.8cm,y=1.8cm,fontsize=\footnotesize]{\tiny Techniques.}
%\Text[x=3.8cm,y=1.5cm,fontsize=\footnotesize]{\tiny (ASAN, CCured,}
%\Text[x=3.8cm,y=1.3cm,fontsize=\footnotesize]{\tiny etc)}

%\Text[x=0.9cm,y=0.6cm,fontsize=\footnotesize]{\bf \faFlagCheckered}

%\Text[x=2.0cm,y=1.4cm,fontsize=\footnotesize]{\bf \footnotesize iC3C}

%\draw [{Implies}-,double,line width=1pt] (1.0,0.8) -- (1.9,2.1);


\draw[-Stealth,thick] (0,0) -- (5.1,0) node[anchor=north west] {\bf \tiny Full};
\Text[x=0cm,y=-0.2cm,fontsize=\footnotesize]{\bf \tiny None}
\Text[x=2.6cm,y=-0.2cm,fontsize=\footnotesize]{Spatial Memory Safety}
\Text[x=-0.2,y=2.6,rotation=90,fontsize=\footnotesize]{Function Risk level}
\draw[-Stealth,thick] (0,0) -- (0,5.1) node[anchor=south east] {\bf \tiny High};
\end{tikzpicture}
\caption{Existing approaches and the desired approach (\faFlagCheckered) along with our proposal (iC3C) to achieve it.}
\label{fig:tainedsplit}
\end{center}
\end{figure}





%\begin{listing*}[t!]
%\inputminted[linenos, mathescape, escapeinside=||, fontsize=\tiny{}]{c}{examples/originalprogram.c}
%\caption{Simple server example.}
%\label{lst:comb}
%\end{listing*}

