\section{Overview and Transcendence}\label{sec:overview}

This section discusses the key merge features between Checked-C and RLBox and the problems we are trying to solve.



\begin{figure}[t]
{\small
  \begin{lstlisting}[xleftmargin=4 mm]
//in checked region

int compare_1(nt_array_ptr<char> x: count (0),
  nt_array_ptr<char> y : count (0)) { 
  int len_x = strlen(x);
  int len_y = strlen(y);
  return sum(x,len_x) < sum(y,len_y);
  }
...

int stringsort(
 nt_array_ptr<nt_array_ptr<char>> s : count (n),
 ptr<(int)(nt_array_ptr<char>,
  nt_array_ptr<char>)> cmp, int n) {
  int i, j, gap;
  int didswap;
 
  for(gap = n / 2; gap > 0; gap /= 2) {
    {
    do  {
        didswap = 0;
        for(i = 0; i < n - gap; i++)
           {
             j = i + gap;
             if((*cmp)(s[i], s[j]) > 0)
             {
                int len = strlen(s[i]);
                nt_array_ptr<char> 
                  tmp : count (len) = s[i];
                s[i] = s[j];
                s[j] = tmp;
                didswap = 1;
             }
           }
        } while(didswap);
    }
  }

  return 0;
}
  \end{lstlisting}
}
\caption{Checked stringsort Code}
\label{fig:checkedc-example-1}
\end{figure}

\begin{figure}[t]
{\small
  \begin{lstlisting}[xleftmargin=4 mm]
//in checked region, tainted version
int tainted_compare_1(
  nt_array_tptr<char> x : count (0),
  nt_array_tptr<char> y : count (0)) {
  int len_x = strlen(x);
  int len_y = strlen(y);
  nt_array_ptr<char> tx : count (len_x)
   = malloc(nt_array<char>, len_x);
  nt_array_ptr<char> ty : count (len_y)
   = malloc(nt_array<char>, len_y);
  safe_memcpy(tx,x,len_x);
  safe_memcpy(ty,y,len_y);
  return compare_1(tx,ty);
}
...

//calling the function turns
//an unchecked region to a checked region.
int tainted_stringsort(nt_array_tptr
     <nt_array_tptr<char>> s : count (n),
 tptr<(int)(nt_array_tptr<char>,
  nt_array_tptr<char>)> cmp, int n) {
  int i;
  nt_array_ptr<nt_array_ptr<char>> p : count (n)
    = malloc(nt_array<nt_array_ptr<char>>, n);
  for(i = 0; i < n; i++) {
    int len = strlen s[i];
    nt_array_ptr<char> tmp : count (len)
      = new malloc(nt_array<char>, len);
    safe_memcpy(tmp,s[i],len);
    p[i] = tmp;
  }
  ptr<(int)(nt_array_ptr<char> : count (0),
  nt_array_ptr<char> : count (0))>
    cfun = find_check(cmp);
  
  return stringsort(p,cfun);
}
  \end{lstlisting}
}
\caption{Tainted stringsort Code}
\label{fig:checkedc-example-2}
\end{figure}

\begin{figure}[t]
{\small
  \begin{lstlisting}[xleftmargin=4 mm]
//in unchecked region
int f(char ** s, int (*cmp)(char *,char *),
  int (*sort)(char **, int (*)(char *,char *),
               int), int n) {
...

  int i = sort(s,cmp,n);
  free(s[10]);
...
}

int g(int (*cmp)(char *,char *)) {
...
  int real_addr = derandomize(cmp);
...
}

int main(int n) {
  nt_array_ptr<nt_array_ptr<char>> p : count(n)
    = malloc(nt_array<nt_array_ptr<char>>, n);

  nt_array_tptr<nt_array_tptr<char>>
     tp : count(n) =
       tmalloc(nt_array<nt_array_tptr<char>>, n);
...
  safe_memcpy(tp, p, n);

  unchecked {
    if (BAD) // a flag to call different funs.
      //input checked pointers
      f(p, compare_1, stringsort); 
    else 
      //input tainted pointers
      f(tp, tainted_compare_1, tainted_stringsort);
  }

  if (!BAD) safe_memcpy(p, tp, n);
  p[10] = "crash?";
  
  unchecked {
    if (BAD) g(compare_1)
      else g(tainted_compare_1);
  }

  return 0;
}
  \end{lstlisting}
}
\caption{Tainted Pointer Usage in Calling Unchecked Fun}
\label{fig:checkedc-example-3}
\end{figure}


\myparagraph{Maintaining Non-crashing}
Previously, the main guarantee of \checkedc \cite{li22checkedc} was the blame theorem,
i.e., if there is a crash, the source is from some unchecked regions.
For example, at \Cref{fig:checkedc-example-3} line 31,
we call a unchecked function \code{f} with a checked null-terminated array (NT-array) pointer argument.
At line 8, depending on the NT-array size, \code{free(s[10])} might crash.
Even if it does not crash, line 38 is doomed because of the \code{free} call.

The philosophy is that unchecked regions are written in C and unstable,
and programmers will eventually remove them by converting them to \checkedc code.
The reality is that users might not want to convert everything.
For some insignificant or handy history C code, such as library functions,
they might want to keep it as long as no crashing.

The sources of crashing in \checkedc are (1) unchecked regions crash themselves;
  and (2) the misuse of checked pointers in unchecked regions.
Enlightened by RLBox \cite{rlbox-paper},
we sandbox the unchecked regions and then we utilize the \checkedc type system
to disallow checked pointers to be used in an unchecked region. 
To achieve the communication between checked and unchecked regions,
we create tainted pointers that can be shared by different regions,
whose data are stored in the sandboxed heap.
Users are required to copy data to tainted pointers that are shared in unchecked regions.
For example,  we copy the checked pointer data to the tainted pointer \code{tp}
at \Cref{fig:checkedc-example-3} line 26,
and input the tainted pointer to the unchecked function at line 34.
At line 8, even if statement might crash, since tainted pointers are stored
in the sandboxed heap, it can be recovered.
At line 37, the use of a tainted pointer in a checked region
requires a verification on it.
This is handled by inserting additional checks
and creating exception handling before the use by the \systemname compiler.
Thus, the checked pointer \code{p} is safely used at line 38.

In short, banning checked pointer uses in unchecked regions and providing tainted shared pointers
are the solution for crashing. 
In \systemname, we proved the non-crashing theorem based on the \systemname type system,
.i.e, any program execution is stopped in a predictable manner
where either the execution terminates or is stopped at a place where an exception handler exists.
In fact, our work is \textit{the first formalism of RLBox},
and formally shows the security guarantee that RLBox provides 
is actually the non-crashing theorem.

\myparagraph{Formalism Function Pointers}
In C, manipulating function pointers is a major way of implementing high order functions.
Previously, we assumed all function calls are through calling by name to a global map.
In \systemname, we formalize function pointers and maintain the \systemname type soundness.
To the best of our knowledge,
this is the first work of formalizing C function pointers with security guarantee.

\Cref{fig:checkedc-example-1} defines a string sorting algorithm
depending on the input function pointer \code{cmp} defining the generic order for strings,
and \code{compare_1} is an example \code{cmp} function
that adds the ASCII numbers of characters in the two strings and compare the results.
In addition, function pointers enable the callback mechanism,
i.e., a server sends a function pointer to a client in an unchecked region,
and allows the client to access some server resources by calling back the pointer.
This is a common usage between a web-browser and untrusted third party libraries.
The function call in \Cref{fig:checkedc-example-3} line 34 is one such usage.

We also utilize \systemname subtyping relation to permit function pointer static auto-casting.
Function pointer type information might contain array pointer bound information,
for which it is inconvenient to coincide the defined types for a function implementation and the function pointer type. 
For example, the \code{cmp} argument in \code{stringsort} (\Cref{fig:checkedc-example-1})
has type \code{ptr<(int)(nt_array_ptr<char>, nt_array_ptr<char>)>},
meaning that the function takes two NT-array pointers with arbitrary size and outputs an integer.
The function \code{compare_1}'s pointer has type \code{ptr<(int)(nt_array_ptr<char> : count (0), nt_array_ptr<char> : count (0) )>}
To use \code{compare_1} in \code{stringsort}, the type is auto-cast to the \code{cmp}'s type.
In general, if function pointer $x$ has type $* (tl \to t)$, and  $y$ has $* (tl' \to t')$,
in order to use $x$ as $y$, $tl'$ should be a subtype of $tl$ and $t$ subtypes to $t'$.

\myparagraph{Not Exposing Checked Pointer Addresses}
The design for the non-crashing property bans the checked pointer uses in unchecked regions.
Thus, there is no reason to permit checked pointer variable assignments in unchecked regions;
especially, this might expose a checked pointer address to untrusted parties.
For example, the call to function $g$ at \Cref{fig:checkedc-example-3} line 41 lives in an unchecked region,
and $g$ might use some mechanism, such as derandomizing ASLR \cite{shacham-aslr},
to achieve the checked pointer address.
Thus, it enables a third party to access any checked heap and function data by simple pointer arithmetic.

To prevent the checked pointer address leak,
we prevent any unchecked regions acknowledge checked pointer variables.
In addition, to facilitate checked function callbacks, 
the \systemname compiler compiles every checked function
with an additional tainted shell function.
Users are required to serve unchecked regions with the tainted shell pointer instead of the original checked function pointer.
For example, \code{tainted_compare_1} and \code{tainted_stringsort} in \Cref{fig:checkedc-example-2} 
are the tainted shells of the checked functions \code{compare_1} and \code{stringsort}.
In the tainted shells, the arguments are tainted versions of the corresponding arguments in the checked functions.
Inside the shell body, create checked pointer copies of the tainted arguments, and call the checked functions.
In addition, once a checked function returns,
if the output is a checked pointer, we also its the data to a new tainted pointer and exit the shell.
\Cref{fig:checkedc-example-3} line 42 is an example of serving the function call living in an unchecked region
with a tainted shell pointer argument \code{tainted_compare_1}.
Even if $g$ derndomizes its address (line 14), the shell address is in the sandbox and has no harm,
and calling the shell never exposes any checked pointer information outside of the shell.

Conceptually, the shell is run in a checked region.
One can imagine that a tainted shell is a safe closure that contains a checked block.
Once the closure is called, the system is turned to a checked region,
so that the checked function call at \Cref{fig:checkedc-example-2} line 13 is safe, 
even if it is called by $g$ in \Cref{fig:checkedc-example-3} , because it lives in the checked region.
In the \systemname formalism, we formalize a \textit{checked} block on top of the existing unchecked regions,
and the transition of a tainted shell call creates a checked block containing the shell body.
We also make sure that no arguments in these tainted shell contains any checked pointers,
as well as no output is of a checked type.

% \paragraph*{\textbf{NT-array Pointer Formalization}} NT-array pointers point to NT-arrays whose endings are determined by a null (\code{'\0'}). The length of a NT-array is usually not fixed. The two uses of NT-array pointers in \checkedc are branching and \code{strlen} operations. \code{if (*x) }$e_1$\code{ else }$e_2$ branches to $e_1$/$e_2$ depending one the \code{x}'s data, while \code{strlen(x)} computes the length of the NT-array pointed to by \code{x}.
% \liyi{move to formal and later section.}
  
% \paragraph*{\textbf{Dependent Functions}} \checkedc allows users to declare dependent functions.  Fig.~\ref{fig:checkedc-example} shows an example usage of dependent functions. The integer argument \code{x} is used to specify the bounds of the two NT-array pointers. It means that when we call the \code{memcpy} function, the upper-bounds of the two NT-array pointers must be no less than \code{x}. If the programmer makes the mistake of writing \lstinline{i <= x} in the for loop, the runtime checks will ensure that the indexing of\lstinline|x| in either \code{a} or \code{b} results in an error.
% Because of the dependent function features, we are able to utilize the fact that the two remaining length of the NT-array pointers are greater than \code{x}, so that the \code{for} is guaranteed to execute properly.

% \ignore{
% \liyi{Good information about pointer types here:
% https://github.com/microsoft/checkedc/wiki/New-pointer-and-array-types}




% This code creates an array pointer of length $10$, and reads the pointer $(x+y)$. 
% If the integer value $y$ is greater than $10$, executing the program in C results in undefined behavior. 
% In \checkedc, however, we links every pointer in a program with their bound information statically, and inserts checks to dynamically verify if the usage of the pointer violates the spatial safety. For example, In compiling the above fragment, we inserts the bound's check for $x$ in the read operation $(x+y)$, and if $y$ results in a value $11$, the execution of the read operation $(x+y)$ results in an dynamic error. 

% There are many existing language development trying to guarantee the C spatial safety property through different approaches. For example, Cyclone \cite{Jim2002} and Cerberus \cite{cerberus} are the fat pointer approaches for guaranteeing the spatial safety and eliminating undefined behaviors in C. Rust \cite{Rust2016}, on the other hand, tries to restrict the possible way of writing programs to guarantee such property statically.
% The problem of these two approaches are effectiveness of executing programs and user practicality of writing programs. The fat-pointer approach completely relies on dynamic checks to guarantee the safety properties but it might pay a huge price on execution efficiency. In our experiment (Sec.~\ref{sec:evaluation}), we found that the execution of fat pointer approaches cost the speed a 50\% overhead. 
% On the other hand, Rust uses a static approach by restricting the usage of pointers to guarantee the properties, which results in a user UN-friendly system.  
% The development of \checkedc is an investigation on keeping the balance between execution efficiency and user-friendly systems. We develop a system to utilize static type systems to allow the \checkedc compiler to insert dynamic checks to guarantee the safety property. The efficiency of \checkedc is in the same level of Rust while we maintain a user-friendly system and inherited most of the operation features from C.

% Another feature of \checkedc is the division of the checked and unchecked blocks. When users try to rewrite their C programs into \checkedc programs, they can develop the code incrementally by rewriting sub-parts of their programs to \checkedc code fragments and placing them as the checked blocks, while keeping the rest C programs as unchecked C programs. \checkedc guarantees that if there is any safety violation, this comes from the unchecked blocks (the blame theory). 

% Here is the contributions of the paper. First, we develop the code formalization of \checkedc and show that the type system in \checkedc is type-sound and satisfies the blame theory with respect to the \checkedc semantics. Specifically, we formalize null-terminated array pointers. To our best knowledge, this is the first work defining such feature in C-related languages. 
% Second, in formalizing the \checkedc language, we investigate the balance between program execution efficiency and user practicality. We re-investigate the \checkedc type system with subtyping relations to make it more efficient and user-friendly.
% Second, we also develop random testing tools in Redex \cite{pltredex} to test the \checkedc compiler and make sure it is properly developed. We are able to find and reproduce many bugs/faults in the \checkedc compiler \liyi{numbers?}. The random testing tool is able to generate tens of thousands of programs to properly validate the \checkedc compiler. Third, we also spent a great amount of work in developing experiments to compare \checkedc and other existing C-like languages that guarantee the spatial safety and eliminating undefined behaviors. We found that \checkedc keeps the best balance between execution efficiency and user practicality.
% }



