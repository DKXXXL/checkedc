\section{Introduction}
Vulnerabilities due to memory corruption, especially spatial memory corruptions, are still a major issue for C programs~\cite{cvetrend, microsoftmemsafe} despite a large body of work that tries
to prevent them~\cite{song2019sanitizing}.
Converting these existing programs into safe languages, such as Java, requires considerable effort and is not (yet) feasible for performance-critical environments such as Embedded Operating Systems.
Although high-performant safe system languages such as Rust~\cite{rustlang} and Go exist, these a too different to constitute a practical target~\cite{zeng2019identifying}, and rewriting in these languages requires considerable effort.
Existing automated conversion tools~\cite{c2rust,c2rusttalk, emre2021c2rust} are able to take only small steps (i.e., to unsafe, non-idiomatic Rust) and still~\emph{require considerable work from developers}.
There are other retrofitting techniques such as CCured~\cite{necula2005ccured},
Softbound~\cite{nagarakatte2009softbound}, Low Fat
pointers~\cite{duck2016heap}, and Address Sanitizer
(ASAN)~\cite{serebryany2012addresssanitizer} aim to enforce spatial safety
automatically~\emph{without any developer effort}, by analyzing a C program and compiling it to include
run-time safety checks. Unfortunately, the resulting~\emph{run-time overhead is too high
for deployment (between 60\%-200\%)}, especially in low-powered IoT devices where the usage of C-based system software (e.g., FreeRTOS) is still rampant.

\noindent
\textbf{Checked C.} Tarditi et al. developed~\emph{Checked C}, a safe dialect of C that prevents spatial memory issues (temporal safety is underway~\cite{checkedc:temporal}) with no overhead.
Checked C extends C with \emph{checked pointer
  types} which are restricted by the compiler to spatially safe uses (temporal safety is underway~\cite{checkedc:temporal}).
Such pointers have one of three possible types, \ptrT{T}, \arrT{T},
or~\ntarrT{T} (\ptr, \arr, and \ntarr{} for short), representing a
pointer to a single element, array of elements, or null-terminated
array of elements of type $T$, respectively. The latter two have an
associated \emph{bounds annotation}; e.g., a declaration \arrT{int}
\code{p} : \code{count(n)} says that \code{p} is a pointer to an \code{int} array whose size is \code{n}.
The compiler uses these bounds annotations to add dynamic checks prior to checked pointer accesses, to prevent spatial safety violations.
These run-time checks can often be proved redundant and removed by LLVM, yielding good performance.
Specifically, Tarditi et al.~\cite{tarditi2018checked} reported average
run-time overheads of 8.6\% on a small benchmark suite, and
Duan et al.~\cite{duanrefactoring} found essentially~\emph{no overhead} when running
Checked C-converted portions of the FreeBSD kernel.
Furthermore, Checked C is backward compatible as the compiler  represents checked pointers as system-level memory words, i.e., without
``fattening'' metadata, ensuring backward compatibility. 
The backward compatibility also enables incremental or partial conversion where Checked annotations can be provided to a few pointers.
Such partially annotated programs still enjoy the benefits of spatial memory safety on those regions using only Checked pointers (i.e., checked or safe regions).
~\fig{fig:memprotections} shows the summary of these existing approaches along with Checked C.
We argue that~\emph{converting to Checked C provides a reasonable and practical approach to adding safety to existing system software written in C}.

\begin{figure}
\begin{center}
\begin{tikzpicture}[scale=0.75]
%\node[text width=3cm] at (1.5,-0.5) 
%    {some text spanning three};
%\draw[step=5cm,gray,very thin] (0,0) grid (5,5);
\fill[red!50!white] (2.5,2.5) rectangle (5,5);
\fill[red!20!white] (0,2.5) rectangle (2.5,5);
\fill[red!20!white] (2.5,0) rectangle (5,2.5);
\fill[green!40!white] (0,0) rectangle (2.5,2.5);

\Text[x=3.8cm,y=4.7cm,fontsize=\tiny]{Managed}
\Text[x=3.8cm,y=4.5cm,fontsize=\tiny]{Languages.}
\Text[x=3.8cm,y=4.3cm,fontsize=\tiny]{(Java, Python,}
\Text[x=3.8cm,y=4.1cm,fontsize=\tiny]{C\#, etc)}

\Text[x=1.2cm,y=4.7cm,fontsize=\tiny]{High Performant}
\Text[x=1.2cm,y=4.5cm,fontsize=\tiny]{Safe Languages.}
\Text[x=1.2cm,y=4.3cm,fontsize=\tiny]{(Rust, Go, etc)}

\Text[x=1.9cm,y=2.3cm,fontsize=\tiny]{Checked C}


\Text[x=3.8cm,y=2.1cm,fontsize=\tiny]{Retrofitting}
\Text[x=3.8cm,y=1.8cm,fontsize=\tiny]{Techniques.}
\Text[x=3.8cm,y=1.5cm,fontsize=\tiny]{(ASAN, CCured,}
\Text[x=3.8cm,y=1.3cm,fontsize=\tiny]{etc)}

\Text[x=0.9cm,y=0.6cm,fontsize=\tiny]{\bf \faFlagCheckered}

\Text[x=2.0cm,y=1.4cm,fontsize=\tiny]{\bf \footnotesize iC3C}

\draw [{Implies}-,double,line width=1pt] (1.0,0.8) -- (1.9,2.1);


\draw[-Stealth,thick] (0,0) -- (5.1,0) node[anchor=north west] {\bf \tiny high};
\Text[x=0cm,y=-0.2cm,fontsize=\tiny]{\bf least}
\Text[x=2.6cm,y=-0.2cm,fontsize=\footnotesize]{Performance Overhead}
\Text[x=-0.2,y=2.6,rotation=90,fontsize=\footnotesize]{Developer effort}
\draw[-Stealth,thick] (0,0) -- (0,5.1) node[anchor=south east] {\bf \tiny high};
\end{tikzpicture}
\caption{Existing approaches to handle memory corruptions vulnerabilities in C programs along with Checked C.}
\label{fig:memprotections}
\end{center}
\end{figure}


\noindent
\textbf{Converting C to Checked C.} The safety guarantees of Checked C come with certain restrictions. For instance, as shown below, Checked C programs cannot use address-taken variables in a bounds expression as the bounds relations may not hold because of possible modifications through pointers.
\begin{minted}[xleftmargin=30pt, mathescape, escapeinside=||, fontsize=\footnotesize]{c}
...
array_ptr<int> p : count (n) = NULL;
|\textcolor{red}{\faTimes}|..,&n,.
\end{minted}
Consequently, converting existing C programs to Checked C might require refactoring, e.g., modifying the above program to not use~\inlinecode{&n} expression, which might require considerable effort~\cite{duanrefactoring} depending on the program's complexity. 
Recently, Machiry et al. developed~\threec~\cite{machiry2022c} that tries to automatically convert a program to Checked C by adding appropriate pointer annotations.
However, as described in \threec, complete automated conversion is infeasible and requires the developer to convert some code regions manually.
Although, the backward compatibility of Checked C helps a partially annotated program to enjoy spatial memory safety on those regions using only Checked pointers (i.e., checked or safe regions).

\noindent
\textbf{No safety against unchecked code.} But, the unconverted code regions (or unsafe regions) can affect pointers in safe regions and violate certain assumptions leading to vulnerabilities, as demonstrated by cross-language attacks~\cite{mergendahlcross}.
Although the blameless proof exists~\cite{ruef2019achieving}, it does not state that spatial safety violations cannot happen in Checked regions but rather states that Checked regions~\emph{cannot be blamed for any spatial safety violations}.
Consider the following example:
\begin{minted}[xleftmargin=25pt, mathescape, linenos, escapeinside=||, fontsize=\footnotesize{}]{c}
// Checked code
int func(array_ptr<char> p : count(5)) {
|\textcolor{red}{\faChainBroken}|..p[4]..
}
// unchecked code
...
str = "he";
...
|\textcolor{red}{\faBug}|assume_bounds_cast<char>(str, 5); 
...
char ptr[16];
...
len <- derived from user input
...
|\textcolor{red}{\faBug}| memcpy(ptr, buff, len); // buffer overflow
\end{minted}
Here, the checked function~\inlinecode{func} expected a pointer to a buffer of five elements, but unchecked code violated it and invoked the function with a buffer of 2 elements.
This results in a spatial safety violation (\textcolor{red}{\faChainBroken}) in the Checked region, but of course, the blame or the root cause is in the unchecked region (\textcolor{red}{\faBug}).
Furthermore, since checked and unchecked regions execute in the same address space, spatial memory corruptions in unchecked regions (Line 15) can take down the complete program despite having checked regions.
We need~\emph{an isolation mechanism to ensure that code executed as part of unchecked regions does not violate the safety guarantees in checked regions}.

This mechanism is called program partitioning~\cite{rul2009towards}, and there has been considerable work~\cite{tan2017principles, brumley2004privtrans, bittau2008wedge, lind2017glamdring, liu2017ptrsplit} in the area. Most of these techniques are~\emph{data-centric}~\cite{lind2017glamdring, liu2017ptrsplit}, wherein program data drives the partitioning. E.g., Given sensitive data in a program, the goal is to partition functions into two parts or partitions based on whether a function can access the sensitive data.
The performance overhead of these approaches is dominated by marshaling costs and depends on the usage of sensitive data.
The overhead of state-of-the-art approaches~\cite{lind2017glamdring, liu2017ptrsplit} is prohibitive and varies from 37\%-163\%.
But in our case, we want a low-overhead code-centric partitioning, where the unchecked code (or functions) should be isolated (or partitioned) from checked code. We also want the technique to co-exist and be compatible with Checked C guarantees such that the partition containing checked code should still enjoy its spatial safety.

In this work, we propose a type-directed code-centric program partitioning approach.
Specifically, our system,~\systemname, extends Checked c using~\textbf{tainted} (\taintt) types, which can be used to mark functions and pointers that need to be isolated from the original program.
The tainted types long with Checked C allow annotating pointer along two dimensions, i.e., (i) taintedness: a pointer can be either tainted or not (untainted), and (ii) checkedness: it can be either checked or not.
The checked types follow the standard Checked C typing rules. However, our type-system disallows all interactions between taint and untainted types.
The developer starts by marking desired (unchecked or risky) functions and pointers used in these functions as tainted.
Second,~\systemname partitions the given program into two partitions (\ucregion and~\cregion)  of different privileges:
\begin{itemize}
\item~\ucregion (low privilege tainted region): This partition contains tainted types (i.e., functions and pointers) and can only access tainted pointers.
\item~\cregion (high privilege untainted region): This partition contains the remaining (untainted) code and data and has complete access to~\cregion{}.
The functions in~\cregion can invoke any function in~\ucregion but not the other way around, except for call-back functions, which we will discuss later.
\end{itemize}
Finally, during program execution, the~\ucregion partition will be executed in an existing sandboxed environment (e.g., WASM sandbox), and our compiler will add the necessary instrumentation to facilitate the communication between code in~\cregion{} and~\ucregion{}.

The combination of tainted types and privileged partitions enables us to enforce isolation and provide memory safety without marshaling costs.
As functions in the~\ucregion can only access tainted types,~\cregion functions should use tainted types to pass pointer arguments to~\ucregion functions. 
We avoid marshaling by allocating all tainted pointers (i.e., tainted buffers) in~\ucregion and thus can be accessed in both partitions.
Although memory isolation prevents direct violations,~\ucregion code can still affect~\cregion through tainted pointers by confused deputy attacks.
Our compiler avoids these attacks by ensuring, either statically or through dynamic checks, that tainted pointers can only point to~\ucregion address space.

%In this work, we propose~\systemname, an isolation mechanism that enables unchecked regions or, in general, any arbitrary set of functions to be executed in an isolated address space (\ucregion).
%So that checked regions can still enjoy true spatial memory safety even in the presence of unchecked or unsafe regions.
%\systemname extends Checked c using~\textbf{tainted} (\taintt) types, representing the code and data that belong to the~\ucregion.
%
%\noindent
%\textbf{Automated Partitioning.} A pointer (checked or unchecked) can be marked as tainted; similarly, a function can be marked as tainted.
%Along with enforcing spatial safety constraints of checked tainted pointers, the~\systemname compiler will bundle all the tainted types (functions and pointers) so that they remain in an isolated sandbox (\ucregion), and non-tainted types will remain in the process address space (\cregion).
%
%\noindent
%\textbf{Memory Access Restrictions.} The functions in the~\ucregion can only access tainted pointers, i.e., memory within the corresponding sandbox region.
%However, the functions in~\cregion have no-such restrictions and can access both c- and uc-region's memory.
%Similarly, the functions in~\cregion can invoke any function in~\ucregion but not the other way around, except for call-back functions, which we will discuss later.
%
%\noindent
%\textbf{Marshaling free communication.} As functions in the~\ucregion can only access tainted types,~\cregion functions should use tainted types to pass pointer arguments to~\ucregion functions. 
%We avoid marshaling as all tainted buffers (i.e., regions pointed by tainted pointers) are allocated in~\ucregion and thus can be accessed in both c- and uc-region.
%
%\noindent
%\textbf{Additional checks.} Although memory isolation prevents direct violations,~\ucregion code can still affect~\cregion through tainted pointers by confused deputy attacks.
%However, our compiler avoids these attacks by ensuring, either statically or through dynamic checks, that tainted pointers in~\cregion can only point to~\ucregion address space, and our type system enforces that tainted pointers cannot be assigned to non-tainted pointers.
%
%%\iffalse
%Recently, Narayan et al. proposed RLBox, which enables sandboxing libraries by making data- and control-flow at the sandbox interface explicit, using types.
%Specifically, RLBox defines a tainted type (\code{tainted<...>}), which is a C++ template class and can be used to have any type tainted (e.g.,~\code{tainted<int>},~\code{tainted<float*>}, etc).
%The tainted types represent values or pointers to buffers that can be accessed from a sandbox.
%The program or unsandboxed code can use only tainted type variables to communicate (i.e., parameters and return values) with sandboxed functions.
%Although RLBox provides a practical technique for sandboxing libraries, it has the following drawbacks:
%\begin{itemize}
%\item\textbf{Marshalling Cost.} Any non-scalar value (e.g., a pointer to a struct) that need to be passed to a sandboxed function should be first copied into a tainted buffer (\emph{marshalling}). Then the corresponding tainted value (i.e., tainted pointer) should be passed to the sandboxed function.
%This can lead to considerable overhead for large objects.
%\item\textbf{Developer Written Sanitization.} The unsandboxed code can access tainted type variables only through certain special (i.e., sanitization) routines.
%This forces developers to sanitize every value that comes from sandboxed library functions.
%However, the correctness of the sanitization is entrusted to the developer.
%Consequently, any mistake in sanitization methods can still lead to vulnerabilities.
%\item\textbf{Library Level Sandboxing.} The unit of sandboxing in RLBox is the whole library. However, we want to sandbox a set of functions in a program.
%\end{itemize}
%\fi

% How to highlighgt code in minted
% highlightlines={1-3}, highlightcolor=red
\ignore{
\begin{listing}[t!]
  \begin{tabular}{c c}
    \begin{minipage}[b]{.22\textwidth}
\inputminted[mathescape, escapeinside=||, fontsize=\tiny{}]{c}{examples/orig1.c}
    \end{minipage} &
    \begin{minipage}[b]{.22\textwidth}
\inputminted[mathescape, escapeinside=||, fontsize=\tiny{}]{c}{examples/orig2.c}
    \end{minipage} \\
   (a) Original C code & (b) After initial conversion.\\
  \end{tabular}
\caption{(Contrived) Example demonstrating various phases of.}
\label{lst:comb}
\end{listing}

\begin{listing}[t!]
\inputminted[mathescape, escapeinside=||, fontsize=\tiny{}]{c}{examples/firstrun.c}
\caption{(Contrived) Example demonstrating various phases of.}
\label{lst:initialconv}
\end{listing}

\begin{listing}[t!]
\inputminted[mathescape, escapeinside=||, fontsize=\tiny{}]{c}{examples/humanannotations.c}
\caption{(Contrived) Example demonstrating various phases of.}
\label{lst:humantaint}
\end{listing}

\begin{listing}[t!]
\inputminted[mathescape, escapeinside=||, fontsize=\tiny{}]{c}{examples/humanadjustments.c}
\caption{(Contrived) Example demonstrating various phases of.}
\label{lst:humanadjust}
\end{listing}
}




%\begin{figure}
\begin{center}
\begin{tikzpicture}[scale=0.75]
%\node[text width=3cm] at (1.5,-0.5) 
%    {some text spanning three};
%\draw[step=5cm,gray,very thin] (0,0) grid (5,5);


% This is all not protected region.
%Top left: High risk and low safety
\fill[red!50!white] (0,2.5) rectangle (2.5,5);
% bottom left: low risk and low safety
\fill[red!20!white] (0,0) rectangle (2.5,2.5);

% This is all protected.
% Top right: High risk and high safety
\fill[green!40!white] (2.5,2.5) rectangle (5,5);
% bottom right: low risk and high safety
\fill[green!40!white] (2.5,0) rectangle (5,2.5);


% High risk, high safety
\Text[x=3.8cm,y=5.3cm,fontsize=\footnotesize]{\tiny Checked C}
\Text[x=3.8cm,y=3.5cm,fontsize=\footnotesize]{\tiny Functions with spatial}
\Text[x=3.8cm,y=3.3cm,fontsize=\footnotesize]{\tiny safety for all}
\Text[x=3.8cm,y=3.1cm,fontsize=\footnotesize]{\tiny pointers.}

\Text[x=1.2cm,y=5.3cm,fontsize=\footnotesize]{\tiny No Safety}
\Text[x=1.2cm,y=4.3cm,fontsize=\footnotesize]{\tiny Sanboxed Functions.}
\Text[x=1.2cm,y=4.1cm,fontsize=\footnotesize]{\tiny (Tainted function)}

\Text[x=1.9cm,y=2.3cm,fontsize=\footnotesize]{\tiny Checked C}


%\Text[x=3.8cm,y=2.1cm,fontsize=\footnotesize]{\tiny Retrofitting}
%\Text[x=3.8cm,y=1.8cm,fontsize=\footnotesize]{\tiny Techniques.}
%\Text[x=3.8cm,y=1.5cm,fontsize=\footnotesize]{\tiny (ASAN, CCured,}
%\Text[x=3.8cm,y=1.3cm,fontsize=\footnotesize]{\tiny etc)}

%\Text[x=0.9cm,y=0.6cm,fontsize=\footnotesize]{\bf \faFlagCheckered}

%\Text[x=2.0cm,y=1.4cm,fontsize=\footnotesize]{\bf \footnotesize iC3C}

%\draw [{Implies}-,double,line width=1pt] (1.0,0.8) -- (1.9,2.1);


\draw[-Stealth,thick] (0,0) -- (5.1,0) node[anchor=north west] {\bf \tiny Full};
\Text[x=0cm,y=-0.2cm,fontsize=\footnotesize]{\bf \tiny None}
\Text[x=2.6cm,y=-0.2cm,fontsize=\footnotesize]{Spatial Memory Safety}
\Text[x=-0.2,y=2.6,rotation=90,fontsize=\footnotesize]{Function Risk level}
\draw[-Stealth,thick] (0,0) -- (0,5.1) node[anchor=south east] {\bf \tiny High};
\end{tikzpicture}
\caption{Existing approaches and the desired approach (\faFlagCheckered) along with our proposal (iC3C) to achieve it.}
\label{fig:tainedsplit}
\end{center}
\end{figure}





%\begin{listing*}[t!]
%\inputminted[linenos, mathescape, escapeinside=||, fontsize=\tiny{}]{c}{examples/originalprogram.c}
%\caption{Simple server example.}
%\label{lst:comb}
%\end{listing*}
